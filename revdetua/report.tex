\documentclass[mirror, portugues]{revdetua}
%
% Valid options are:
%   portugues --------- main language is Portuguese
%   final ------------- final version (default)
%   times ------------- use times (postscript) fonts for text
%   mirror ------------ prints a mirror image of the paper (with dvips)
%   visiblelabels ----- \SL, \SN, \SP, \EL, \EN, etc. defined
%   invisiblelabels --- \SL, \SN, \SP, \EL, \EN, etc. not defined (default)
%
% Note: the final version should use the times fonts
% Note: the really final version should also use the mirror option
%

\usepackage[portuguese]{babel}
\usepackage[utf8]{inputenc}
\usepackage{amsmath} 
\usepackage{comment}
\usepackage{algorithm}
\usepackage{algpseudocode}
\floatname{algorithm}{Algoritmo}
\usepackage{graphicx}
\usepackage[justification=centering]{caption}
\usepackage{float}
%-------------------------------------
% compiling:
% Recipe: xelatex
% Recipe: pdflatex -> bibtex -> pdflatex -> pdflatex
% Recipe: xelatex
%
% notas:
% rever se algorimtos e imagens estão onde devem
%-------------------------------------
\begin{document}

\Header{02}{3}{Dezembro}{2024}{1}

\title{Maximum Weight Cut Problem}
% MUDAR TITULO ALLEZ ALLEZ ALGO COM RANDOMIZADO
\author{Hugo Veríssimo - 124348 - hugoverissimo@ua.pt}
\maketitle

\begin{abstract}
... abstrato em ingles
\end{abstract}

\begin{resumo}
Este relatório apresenta a implementação e comparação de dois métodos para resolver o problema \textit{Maximum Weight Cut}: uma pesquisa exaustiva e uma heurística gulosa. O problema \textit{Maximum Weight Cut} con ESTE É O ANTIGO FAZER NOVO
\end{resumo}

\section{Introdução}

ja se analisou no outro relatorio a descrição do problema \textit{Maximum Weight Cut}, \cite{BS22} e ns q, super fixe

este relatoria visa explorar algoritmos com um certo grau de estocacidade/aletorieda com vista em otimizar a complexidade e as solucoes.

para alem disso os resultados são comparados aos obtidos anteriormente

serao entao implexmentados 3 algoritmos, nomeadamente: ... e ...

\section{Metodologia da Análise}

vamos usar o python por ter o modulo random e outros

ns q vamos usar os ficheiro tal e tal 

e para testar os algortimos serão testados os graficos do Gset e criados por nós com o ficheiro tal

Graphs for the Computational Experiments: mine and elearnig ou links and gset

\section{Algoritmo de 1}

- falar de como sao construidos: componente aletoria e determinisica ?

- Ensuring that no such solutions are tested more than once., como fiz isto

- quando é q o algortimo para?

% algortimo de pesquisa exuastiva so apra referencia

\begin{algorithm}[H]
    \raggedright
    \textbf{Entrada:} matriz de adjacência \textit{G} \\
    \textbf{Saída:} subconjuntos \textit{S} e \textit{T}, peso do corte \textit{weight} \\
    \hrule 
    \caption{Pesquisa Exaustiva}
    \begin{algorithmic}[1]
        \State input\_set $\gets$ \{0, 1, \ldots, \text{len(G)} - 1\}
        \State subsets $\gets$ \text{EMPTY LIST}
        \State n $\gets$ \text{LENGTH OF input\_set}
        
        \Comment{Generate all subsets}
        \For{r from 0 to n}
            \For{each S in combinations(input\_set, r)}
            \State Add S to subsets
            \EndFor
        \EndFor
        
        \State best $\gets$ input\_set
        \State weight $\gets$ 0
        
        \Comment{Evaluate each subset}
        \For{each S in subsets}
            \State new\_weight $\gets$ 0
            \For{each i in S}
                \For{each j in input\_set - S}
                    \State new\_weight $\gets$ new\_weight + G[i, j]
                \EndFor
            \EndFor
            
            \If{new\_weight $>$ weight}
            \State best $\gets$ S
            \State weight $\gets$ new\_weight
            \EndIf
        \EndFor
        \State S $\gets$ best
        \State T $\gets$ input\_set - best
        \State \Return S, T, weight
    \end{algorithmic}
\end{algorithm}

    


- complexidade

\section{Algoritmo de 2}

...

- complexidade

- falar de como sao construidos: componente aletoria e determinisica ?

- Ensuring that no such solutions are tested more than once., como fiz isto

- quando é q o algortimo para?


\section{Algoritmo de 3}

...

- complexidade

- falar de como sao construidos: componente aletoria e determinisica ?

- Ensuring that no such solutions are tested more than once., como fiz isto

- quando é q o algortimo para?

\section{Análise dos Resultados}

Compare the results of the experimental and the formal analysis.

todos os grafos devem ser corridos pelo menos 5 vezes, e a media dos resultados deve ser calculada e mediana do tempo , por causa dos tempos e da aleatoriedade dos resultados

Graphs for the Computational Experiments: mine and elearnig and gset

asdasds

\subsection{(1) the number of basic operations carried out}

dsadasds

\subsection{2 the execution time }

- Determine the largest graph that you can process on your computer, without taking too much time.

- Estimate the execution time that would be required by much larger problem instances.

dsadasd

\subsection{solution}

asdad

\subsubsection{(3) the number of solutions / configurations tested}

sadsad

\subsubsection{precision}

asdasd



\bibliography{refs}

\end{document}
